% !TeX encoding = UTF-8

\thispagestyle{empty}

\begin{center}

\vspace*{-10mm}

{\LARGE INSTITUTE OF\\COMPUTER SCIENCE\\[1mm]}
FREIE UNIVERSITÄT BERLIN\\

\vspace*{0.5cm}

\includegraphics[width=0.18\textwidth]{fu_logo}

\vspace*{1.0cm}

{\Large \textbf{Bachelor Thesis}}\\ 

\vspace{0.2cm}

\LARGE{Lexicographic Fréchet Matchings with Degenerate Inputs}

\vspace{0.6cm}

{\large Anton Irmfried Norbert Begehr}\\
\normalsize
Matriculation-Nr: 5013449\\
a.begehr@fu-berlin.de

\vspace{0.4cm}

\begin{tabular}{rl}
Supervisor: & Prof. Dr. Günter Rote\\
\end{tabular}

\vspace{0.4cm}

Berlin, \today

\vfill

\begin{abstract}
   The classical Fréchet distance is the minimal connecting distance needed to monotonically traverse two paths completely. Lexicographic Fréchet matchings utilise the steepest descent and apply the Fréchet distance recursively (i.e.\ divide and conquer) to minimise the time spent at a distance greater than a threshold. We take a deep dive into degenerate inputs where different matchings appear equal when viewed at a distance. The problem of choosing one of the multiple paths through critical events at the same distance is investigated. Multiple critical events are compared using derivatives and second derivatives.
\end{abstract}

\end{center}
